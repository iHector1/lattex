%% page2.tex
%% Página de Llantas y Frenos — 4 secciones simples (sin cuadros)

\vspace*{1mm}

% -- Título principal (mismo estilo que "Líquidos") --
\noindent\makebox[\textwidth][c]{%
  \begin{tcolorbox}[
    colback=gray!25,
    colframe=gray!25,
    width=\contentwidth,
    arc=1mm,
    boxrule=0pt,
    left=\padlr,
    right=\padlr,
    top=1mm,
    bottom=1mm
  ]
    \centering
    {\fontsize{15}{18}\selectfont \bfseries Llantas y Frenos}
  \end{tcolorbox}%
}

\vspace{2mm}

% -- Estilo visual limpio para la tabla --
\newcommand{\FieldH}{3.4ex}

% -- Control de bordeado (ajusta estos valores) --
\newcommand{\WheelOuterBorderColor}{gray!70}
\newcommand{\WheelOuterBorderWidth}{0.8pt}
\newcommand{\WheelOuterBorderRadius}{1.2mm}
\newcommand{\WheelCellBorderColor}{black!35}
\newcommand{\WheelCellBorderWidth}{0.55pt}
\newcommand{\WheelCellBorderRadius}{1.0mm}

\newcommand{\WheelHeaderBrandBG}{blue!85!black}
\newcommand{\WheelHeaderModelBG}{gray!20}
\newcommand{\WheelHeaderBorderColor}{gray!55}
\newcommand{\WheelHeaderBorderWidth}{0.45pt}
\newcommand{\WheelHeaderHeight}{8.8mm}
\newcommand{\WheelLogoWidth}{0.88\linewidth}
\newcommand{\WheelLogoHeight}{5.2mm}
\newcommand{\WheelCarFile}{carro.png}
\newcommand{\WheelCarWidth}{0.20\textwidth}
\newcommand{\WheelCarTopSpace}{1.4mm}
\newcommand{\WheelCarBottomSpace}{1.4mm}

\newcommand{\StaticFieldColor}[2]{%
  \begin{tcolorbox}[enhanced,nobeforeafter,
    colback=#2,colframe=\WheelCellBorderColor,
    boxrule=\WheelCellBorderWidth,arc=\WheelCellBorderRadius,
    width=\linewidth,
    boxsep=0mm,
    left=0mm,right=0mm,top=1.0mm,bottom=1.0mm,
    valign=center,halign=center
  ]
    {\fontsize{8.4}{9.8}\selectfont\bfseries #1}
  \end{tcolorbox}%
}

\newcommand{\ValueFieldColor}[2]{%
  \begin{tcolorbox}[enhanced,nobeforeafter,
    colback=#2,colframe=\WheelCellBorderColor,
    boxrule=\WheelCellBorderWidth,arc=\WheelCellBorderRadius,
    width=\linewidth,
    boxsep=0mm,
    left=0mm,right=0mm,top=1.0mm,bottom=1.0mm,
    valign=center,halign=center
  ]
    {\fontsize{8.4}{9.8}\selectfont #1}
  \end{tcolorbox}%
}

% -- Personalización rápida (marca y modelo de llanta) --
% Cambia grosor/radio/color de bordes arriba en la sección "Control de bordeado"
% Cambia este archivo para usar otro logo (ruta relativa o absoluta al proyecto)
\newcommand{\WheelLogoFile}{Goodyear.png}
% Cambia este texto para el nombre/modelo de la llanta
\newcommand{\WheelModelName}{WRANGLER ALL TERRAIN ADVENTURE W/KEVLAR 110T}

% -- Bloque simple editable por posición (sin contenedor) --
% #1 Posición | #2 Logo | #3 Nombre llanta | #4 Medida | #5 PSI | #6 Profundidad | #7 Desgaste discos | #8 Desgaste balatas
\newcommand{\SimpleWheelBlock}[8]{%
  {\centering\bfseries\kodemono #1\par}
  \vspace{0.8mm}

  \begin{tcolorbox}[enhanced,nobeforeafter,
    width=\linewidth,
    colback=white,
    colframe=\WheelOuterBorderColor,
    boxrule=\WheelOuterBorderWidth,
    arc=\WheelOuterBorderRadius,
    boxsep=0mm,
    left=0mm,right=0mm,top=0mm,bottom=0mm
  ]
    % Encabezado estilo barra: marca (izquierda) + modelo (derecha)
    \setlength{\tabcolsep}{0pt}
    \begin{tabular}{@{}p{0.24\linewidth}p{0.76\linewidth}@{}}
      \begin{tcolorbox}[enhanced,nobeforeafter,
        width=\linewidth,height=\WheelHeaderHeight,
        colback=\WheelHeaderBrandBG,colframe=\WheelHeaderBorderColor,
        boxrule=\WheelHeaderBorderWidth,arc=0mm,
        boxsep=0mm,left=0.6mm,right=0.6mm,top=0mm,bottom=0mm,
        valign=center,halign=flush left
      ]
        \includegraphics[width=\WheelLogoWidth,height=\WheelLogoHeight,keepaspectratio]{#2}
      \end{tcolorbox}
      &
      \begin{tcolorbox}[enhanced,nobeforeafter,
        width=\linewidth,height=\WheelHeaderHeight,
        colback=\WheelHeaderModelBG,colframe=\WheelHeaderBorderColor,
        boxrule=\WheelHeaderBorderWidth,arc=0mm,
        boxsep=0mm,left=1.0mm,right=1.0mm,top=0mm,bottom=0mm,
        valign=center,halign=flush left
      ]
        {\fontsize{6.6}{7.6}\selectfont\bfseries #3}
      \end{tcolorbox}
      \\
    \end{tabular}

    \vspace{0.8mm}

    % Tabla limpia: cada fila con color propio (sin bordes de color)
    \setlength{\tabcolsep}{0pt}
    \renewcommand{\arraystretch}{1.0}
    \begin{tabular}{@{}p{0.50\linewidth}p{0.50\linewidth}@{}}
      \StaticFieldColor{MEDIDA}{warningyellow} & \ValueFieldColor{#4}{gray!15} \\
      \StaticFieldColor{PSI}{warningyellow} & \ValueFieldColor{#5}{gray!15} \\
      \StaticFieldColor{PROFUNDIDAD}{warningyellow} & \ValueFieldColor{#6}{gray!15} \\
      \StaticFieldColor{DESGASTE DISCOS}{warningyellow} & \ValueFieldColor{#7}{gray!15} \\
      \StaticFieldColor{DESGASTE BALATAS}{warningyellow} & \ValueFieldColor{#8}{gray!15} \\
    \end{tabular}
  \end{tcolorbox}
}

% -- Distribución en 4 partes --
\noindent\makebox[\textwidth][c]{%
\begin{minipage}{\contentwidth}

  \begin{minipage}[t]{0.49\linewidth}
    \SimpleWheelBlock{Delantera Izquierda}{\WheelLogoFile}{\WheelModelName}{315/35/R21}{25}{3 mm}{9 mm}{8 mm}
  \end{minipage}
  \hfill
  \begin{minipage}[t]{0.49\linewidth}
    \SimpleWheelBlock{Delantera Derecha}{\WheelLogoFile}{\WheelModelName}{315/35/R21}{25}{3 mm}{9 mm}{8 mm}
  \end{minipage}

  \vspace{\WheelCarTopSpace}
  {\centering\includegraphics[width=\WheelCarWidth,keepaspectratio]{\WheelCarFile}\par}
  \vspace{\WheelCarBottomSpace}

  \begin{minipage}[t]{0.49\linewidth}
    \SimpleWheelBlock{Trasera Izquierda}{\WheelLogoFile}{\WheelModelName}{315/35/R21}{25}{3 mm}{9 mm}{8 mm}
  \end{minipage}
  \hfill
  \begin{minipage}[t]{0.49\linewidth}
    \SimpleWheelBlock{Trasera Derecha}{\WheelLogoFile}{\WheelModelName}{315/35/R21}{25}{3 mm}{9 mm}{8 mm}
  \end{minipage}

\end{minipage}%
}
