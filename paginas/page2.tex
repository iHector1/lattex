%% page2.tex
%% Página de Llantas y Frenos — 4 secciones simples (sin cuadros)

\vspace*{1mm}

% -- Título principal (mismo estilo que "Líquidos") --
\noindent\makebox[\textwidth][c]{%
  \begin{tcolorbox}[
    colback=gray!25,
    colframe=gray!25,
    width=\contentwidth,
    arc=1mm,
    boxrule=0pt,
    left=\padlr,
    right=\padlr,
    top=1mm,
    bottom=1mm
  ]
    \centering
    {\fontsize{15}{18}\selectfont \bfseries Llantas y Frenos}
  \end{tcolorbox}%
}

\vspace{2mm}

% -- Estilo visual limpio para la tabla --
\newcommand{\FieldH}{3.4ex}

% =====================================================================
% -- Control de bordeado --
% =====================================================================
\newcommand{\WheelOuterBorderColor}{gray!70}
\newcommand{\WheelOuterBorderWidth}{0.8pt}
\newcommand{\WheelOuterBorderRadius}{1.2mm}
\newcommand{\WheelCellBorderColor}{black!35}
\newcommand{\WheelCellBorderWidth}{0pt}
\newcommand{\WheelCellBorderRadius}{1.0mm}

\newcommand{\WheelHeaderBrandBG}{white}
\newcommand{\WheelHeaderModelBG}{gray!20}
\newcommand{\WheelHeaderBorderColor}{gray!55}
\newcommand{\WheelHeaderBorderWidth}{0pt}
\newcommand{\WheelHeaderHeight}{4.2mm}
\newcommand{\WheelInfoFontSize}{4.8}
\newcommand{\WheelInfoFontLead}{5.8}
\newcommand{\WheelMetricTitleSize}{5.6}
\newcommand{\WheelMetricTitleLead}{6.4}
\newcommand{\WheelMetricValueSize}{6.2}
\newcommand{\WheelMetricValueLead}{7.2}
\newcommand{\WheelMetricTitleH}{6.1mm}
\newcommand{\WheelMetricValueH}{5.8mm}
\newcommand{\WheelMetricGap}{0.5mm}
\newcommand{\WheelMetricRowGap}{0.9mm}
\newcommand{\WheelMetricColGap}{0.9mm}
\newcommand{\WheelBrakeHeaderBG}{gray!25}
\newcommand{\WheelBrakeHeaderTextSize}{8.0}
\newcommand{\WheelBrakeHeaderTextLead}{9.2}
\newcommand{\WheelBrakeHeaderH}{5.1mm}
\newcommand{\WheelBrakeRowH}{5.8mm}
\newcommand{\WheelBrakeFontSize}{6.0}
\newcommand{\WheelBrakeFontLead}{7.0}
\newcommand{\WheelBrakeGap}{0.8mm}
\newcommand{\WheelBrakeRowWidth}{0.96\linewidth}
\newcommand{\WheelBrakeColA}{0.31}
\newcommand{\WheelBrakeColB}{0.28}
\newcommand{\WheelBrakeColC}{0.20}
\newcommand{\WheelBrakeColD}{0.14}
\newcommand{\WheelBrakeYellowBG}{warningyellow}
\newcommand{\WheelBrakeGrayBG}{gray!15}
\newcommand{\WheelBrakeGreenText}{green!55!black}

% =====================================================================
% -- Logo Goodyear
%    Para hacerlo MÁS GRANDE: aumenta \WheelLogoColRatio (máx ~0.38)
%    y reduce \WheelModelColRatio en la misma cantidad.
%    Suma debe ser siempre = 1.00
%
%    Logo col : 0.32  (era 0.24 original — ahora más ancho)
%    Modelo col: 0.68
% =====================================================================
\newcommand{\WheelLogoColRatio}{0.32}    % <-- AQUÍ controlas el tamaño del logo
\newcommand{\WheelModelColRatio}{0.68}   % <-- complementario (1 - WheelLogoColRatio)

\newcommand{\WheelLogoMaxWidth}{\linewidth}   % ocupa todo el ancho de su celda
\newcommand{\WheelLogoMaxHeight}{\WheelHeaderHeight}  % ocupa todo el alto del encabezado

% =====================================================================
% -- Carro (valores originales) --
% =====================================================================
\newcommand{\WheelCarFile}{carro.png}
\newcommand{\WheelCarWidth}{.96\linewidth}
\newcommand{\WheelCarTopSpace}{1.4mm}
\newcommand{\WheelCarBottomSpace}{1.4mm}

% =====================================================================
% -- Posición vertical --
% =====================================================================
\newcommand{\WheelSideLift}{-3mm}
\newcommand{\WheelCarDrop}{-3mm}

% -- Layout horizontal (valores originales del carro) --
\newcommand{\WheelSideColWidth}{0.39\linewidth}
\newcommand{\WheelCarColWidth}{0.16\linewidth}
\newcommand{\WheelGapLeftToCar}{0.03\linewidth}
\newcommand{\WheelGapCarToRight}{0.03\linewidth}
\newcommand{\WheelCardRaise}{\WheelSideLift}
\newcommand{\WheelCarRaise}{\WheelCarDrop}

% =====================================================================
% -- Comandos de celda --
% =====================================================================
\newcommand{\WheelMetricCard}[2]{%
  \begin{tcolorbox}[enhanced,nobeforeafter,
    colback=warningyellow,colframe=warningyellow,
    boxrule=0pt,arc=1.2mm,
    width=\linewidth,height=\WheelMetricTitleH,
    boxsep=0mm,
    left=0.8mm,right=0.8mm,top=0mm,bottom=0mm,
    valign=center,halign=center
  ]
    {\fontsize{\WheelMetricTitleSize}{\WheelMetricTitleLead}\selectfont\bfseries\centering #1}
  \end{tcolorbox}%
  \vspace{\WheelMetricGap}
  \begin{tcolorbox}[enhanced,nobeforeafter,
    colback=gray!15,colframe=gray!15,
    boxrule=0pt,arc=1.2mm,
    width=\linewidth,height=\WheelMetricValueH,
    boxsep=0mm,
    left=0.8mm,right=0.8mm,top=0mm,bottom=0mm,
    valign=center,halign=center
  ]
    {\fontsize{\WheelMetricValueSize}{\WheelMetricValueLead}\selectfont\bfseries\centering #2}
  \end{tcolorbox}%
}

\newcommand{\WheelBrakeRow}[4]{%
  \noindent\tcbox[on line,enhanced,nobeforeafter,
    colback=\WheelBrakeYellowBG,colframe=\WheelBrakeYellowBG,
    boxrule=0pt,arc=1.1mm,
    width=\WheelBrakeColA\WheelBrakeRowWidth,height=\WheelBrakeRowH,
    boxsep=0mm,left=0.8mm,right=0.8mm,top=0mm,bottom=0mm,
    valign=center,halign=center
  ]{{\fontsize{\WheelBrakeFontSize}{\WheelBrakeFontLead}\selectfont\bfseries #1}}%
  \hspace{\WheelBrakeGap}%
  \tcbox[on line,enhanced,nobeforeafter,
    colback=\WheelBrakeGrayBG,colframe=\WheelBrakeGrayBG,
    boxrule=0pt,arc=1.1mm,
    width=\WheelBrakeColB\WheelBrakeRowWidth,height=\WheelBrakeRowH,
    boxsep=0mm,left=0.8mm,right=0.8mm,top=0mm,bottom=0mm,
    valign=center,halign=center
  ]{{\fontsize{\WheelBrakeFontSize}{\WheelBrakeFontLead}\selectfont\bfseries\color{\WheelBrakeGreenText} #2}}%
  \hspace{\WheelBrakeGap}%
  \tcbox[on line,enhanced,nobeforeafter,
    colback=\WheelBrakeYellowBG,colframe=\WheelBrakeYellowBG,
    boxrule=0pt,arc=1.1mm,
    width=\WheelBrakeColC\WheelBrakeRowWidth,height=\WheelBrakeRowH,
    boxsep=0mm,left=0.8mm,right=0.8mm,top=0mm,bottom=0mm,
    valign=center,halign=center
  ]{{\fontsize{\WheelBrakeFontSize}{\WheelBrakeFontLead}\selectfont\bfseries #3}}%
  \hspace{\WheelBrakeGap}%
  \tcbox[on line,enhanced,nobeforeafter,
    colback=\WheelBrakeGrayBG,colframe=\WheelBrakeGrayBG,
    boxrule=0pt,arc=1.1mm,
    width=\WheelBrakeColD\WheelBrakeRowWidth,height=\WheelBrakeRowH,
    boxsep=0mm,left=0.8mm,right=0.8mm,top=0mm,bottom=0mm,
    valign=center,halign=center
  ]{{\fontsize{\WheelBrakeFontSize}{\WheelBrakeFontLead}\selectfont\bfseries\color{\WheelBrakeGreenText} #4}}%
}

% -- Llanta y modelo --
\newcommand{\WheelLogoFile}{Goodyear.png}
\newcommand{\WheelModelName}{WRANGLER ALL TERRAIN ADVENTURE W/KEVLAR 110T}

% =====================================================================
% -- Bloque de rueda
% =====================================================================
\newcommand{\SimpleWheelBlock}[8]{%
  {\centering\bfseries\kodemono #1\par}
  \vspace{0.8mm}

  \begin{tcolorbox}[enhanced,nobeforeafter,
    width=\linewidth,
    colback=white,
    colframe=\WheelOuterBorderColor,
    boxrule=\WheelOuterBorderWidth,
    arc=\WheelOuterBorderRadius,
    boxsep=0mm,
    left=0mm,right=0mm,top=0mm,bottom=0mm
  ]
    % -- Encabezado: logo | modelo --
    \setlength{\tabcolsep}{0pt}
    \begin{tabular}{@{}p{\WheelLogoColRatio\linewidth}p{\WheelModelColRatio\linewidth}@{}}
      % Celda logo: sin padding para que la imagen llene toda la celda
      \begin{tcolorbox}[enhanced,nobeforeafter,
        width=\linewidth,height=\WheelHeaderHeight,
        colback=\WheelHeaderBrandBG,colframe=\WheelHeaderBorderColor,
        boxrule=\WheelHeaderBorderWidth,arc=0mm,
        boxsep=0mm,left=0mm,right=0mm,top=0mm,bottom=0mm,
        valign=center,halign=center
      ]
        \includegraphics[width=\WheelLogoMaxWidth,height=\WheelLogoMaxHeight,keepaspectratio]{#2}
      \end{tcolorbox}
      &
      \begin{tcolorbox}[enhanced,nobeforeafter,
        width=\linewidth,height=\WheelHeaderHeight,
        colback=\WheelHeaderModelBG,colframe=\WheelHeaderBorderColor,
        boxrule=\WheelHeaderBorderWidth,arc=0mm,
        boxsep=0mm,left=1.0mm,right=1.0mm,top=0mm,bottom=0mm,
        valign=center,halign=flush left
      ]
        {\fontsize{\WheelInfoFontSize}{\WheelInfoFontLead}\selectfont #3}
      \end{tcolorbox}
      \\
    \end{tabular}

    \vspace{0.8mm}

    % -- Datos de la llanta (estilo tarjeta con título/valor) --
    \setlength{\tabcolsep}{0pt}
    \begin{tabular}{@{}p{0.32\linewidth}@{\hspace{\WheelMetricColGap}}p{0.32\linewidth}@{\hspace{\WheelMetricColGap}}p{0.32\linewidth}@{}}
      \WheelMetricCard{MEDIDA}{#4} &
      \WheelMetricCard{PSI}{#5} &
      \WheelMetricCard{PROFUNDIDAD}{#6} \\
    \end{tabular}

    \vspace{\WheelMetricRowGap}

    \begin{tcolorbox}[enhanced,nobeforeafter,
      colback=\WheelBrakeHeaderBG,colframe=\WheelBrakeHeaderBG,
      boxrule=0pt,arc=0mm,
      width=\linewidth,height=\WheelBrakeHeaderH,
      boxsep=0mm,left=0mm,right=0mm,top=0mm,bottom=0mm,
      valign=center,halign=center
    ]
      {\fontsize{\WheelBrakeHeaderTextSize}{\WheelBrakeHeaderTextLead}\selectfont\bfseries FRENOS, DISCOS Y TAMBORES}
    \end{tcolorbox}

    \vspace{\WheelMetricRowGap}

    \noindent\makebox[\linewidth][c]{\WheelBrakeRow{Balatas/Tambores}{BUENA (>9MM)}{Desgaste}{#8}}%

    \vspace{\WheelMetricRowGap}

    \noindent\makebox[\linewidth][c]{\WheelBrakeRow{Discos}{BUENA (>9MM)}{Desgaste}{#7}}%
  \end{tcolorbox}
}

% =====================================================================
% -- Layout 3 columnas: tarjetas | carro | tarjetas
% =====================================================================
\noindent\makebox[\textwidth][c]{%
\begin{minipage}{\contentwidth}

  \raisebox{\WheelCardRaise}[0pt][0pt]{%
  \begin{minipage}[t]{\WheelSideColWidth}
    \SimpleWheelBlock{Delantera Derecha}{\WheelLogoFile}{\WheelModelName}{315/35/R21}{25}{3 mm}{9 mm}{8 mm}
    \vspace{2.2mm}
    \SimpleWheelBlock{Trasera Derecha}{\WheelLogoFile}{\WheelModelName}{315/35/R21}{25}{3 mm}{9 mm}{8 mm}
  \end{minipage}%
  }%
  \hspace*{\WheelGapLeftToCar}%
  \raisebox{\WheelCarRaise}[0pt][0pt]{%
  \begin{minipage}[t]{\WheelCarColWidth}
    \vspace{\WheelCarTopSpace}
    {\centering\includegraphics[width=\WheelCarWidth,keepaspectratio]{\WheelCarFile}\par}
    \vspace{\WheelCarBottomSpace}
  \end{minipage}%
  }%
  \hspace*{\WheelGapCarToRight}%
  \raisebox{\WheelCardRaise}[0pt][0pt]{%
  \begin{minipage}[t]{\WheelSideColWidth}
    \SimpleWheelBlock{Delantera Izquierda}{\WheelLogoFile}{\WheelModelName}{315/35/R21}{25}{3 mm}{9 mm}{8 mm}
    \vspace{2.2mm}
    \SimpleWheelBlock{Trasera Izquierda}{\WheelLogoFile}{\WheelModelName}{315/35/R21}{25}{3 mm}{9 mm}{8 mm}
  \end{minipage}%
  }

\end{minipage}%
}
