%% page2.tex
%% Página de Llantas y Frenos — 4 secciones simples (sin cuadros)

\vspace*{1mm}

% -- Título principal (mismo estilo que "Líquidos") --
\noindent\makebox[\textwidth][c]{%
  \begin{tcolorbox}[
    colback=gray!25,
    colframe=gray!25,
    width=\contentwidth,
    arc=1mm,
    boxrule=0pt,
    left=\padlr,
    right=\padlr,
    top=1mm,
    bottom=1mm
  ]
    \centering
    {\fontsize{15}{18}\selectfont \bfseries Llantas y Frenos}
  \end{tcolorbox}%
}

\vspace{2mm}

% -- Estilo visual limpio para la tabla --
\newcommand{\FieldH}{3.4ex}

% =====================================================================
% -- Control de bordeado --
% =====================================================================
\newcommand{\WheelOuterBorderColor}{gray!70}
\newcommand{\WheelOuterBorderWidth}{0.8pt}
\newcommand{\WheelOuterBorderRadius}{1.2mm}
\newcommand{\WheelCellBorderColor}{black!35}
\newcommand{\WheelCellBorderWidth}{0pt}
\newcommand{\WheelCellBorderRadius}{1.0mm}

\newcommand{\WheelHeaderBrandBG}{white}
\newcommand{\WheelHeaderModelBG}{gray!20}
\newcommand{\WheelHeaderBorderColor}{gray!55}
\newcommand{\WheelHeaderBorderWidth}{0pt}
\newcommand{\WheelHeaderHeight}{4.2mm}
\newcommand{\WheelInfoFontSize}{4.8}
\newcommand{\WheelInfoFontLead}{5.8}
\newcommand{\WheelMetricTitleSize}{\WheelInfoFontSize}
\newcommand{\WheelMetricTitleLead}{\WheelInfoFontLead}
\newcommand{\WheelMetricValueSize}{\WheelInfoFontSize}
\newcommand{\WheelMetricValueLead}{\WheelInfoFontLead}
\newcommand{\WheelMetricTitleH}{6.1mm}
\newcommand{\WheelMetricValueH}{5.8mm}
\newcommand{\WheelMetricGap}{0.5mm}
\newcommand{\WheelMetricRowGap}{0.9mm}
\newcommand{\WheelMetricColGap}{0.9mm}
\newcommand{\WheelMetricFiveColW}{0.172\linewidth}
\newcommand{\WheelMetricDepthColW}{0.212\linewidth}
\newcommand{\WheelMetricSidePad}{1.0mm}
\newcommand{\WheelBrakeHeaderBG}{gray!25}
\newcommand{\WheelBrakeHeaderTextSize}{\WheelInfoFontSize}
\newcommand{\WheelBrakeHeaderTextLead}{\WheelInfoFontLead}
\newcommand{\WheelBrakeHeaderH}{5.1mm}
\newcommand{\WheelBrakeRowH}{5.8mm}
\newcommand{\WheelBrakeFontSize}{\WheelInfoFontSize}
\newcommand{\WheelBrakeFontLead}{\WheelInfoFontLead}
\newcommand{\WheelBrakeGap}{0.8mm}
\newcommand{\WheelBrakeColA}{0.30\linewidth}
\newcommand{\WheelBrakeColB}{0.29\linewidth}
\newcommand{\WheelBrakeColC}{0.20\linewidth}
\newcommand{\WheelBrakeColD}{0.14\linewidth}
\newcommand{\WheelBrakeYellowBG}{warningyellow}
\newcommand{\WheelBrakeGrayBG}{gray!15}
\newcommand{\WheelBrakeGreenText}{green!55!black}

% =====================================================================
% -- Logo Goodyear
%    Para hacerlo MÁS GRANDE: aumenta \WheelLogoColRatio (máx ~0.38)
%    y reduce \WheelModelColRatio en la misma cantidad.
%    Suma debe ser siempre = 1.00
%
%    Logo col : 0.32  (era 0.24 original — ahora más ancho)
%    Modelo col: 0.68
% =====================================================================
\newcommand{\WheelLogoColRatio}{0.32}    % <-- AQUÍ controlas el tamaño del logo
\newcommand{\WheelModelColRatio}{0.68}   % <-- complementario (1 - WheelLogoColRatio)

\newcommand{\WheelLogoMaxWidth}{\linewidth}   % ocupa todo el ancho de su celda
\newcommand{\WheelLogoMaxHeight}{\WheelHeaderHeight}  % ocupa todo el alto del encabezado

% =====================================================================
% -- Carro (valores originales) --
% =====================================================================
\newcommand{\WheelCarFile}{carro.png}
\newcommand{\WheelCarWidth}{1.34\linewidth}
\newcommand{\WheelCarShiftX}{-2.2mm}
\newcommand{\WheelCarTopSpace}{1.4mm}
\newcommand{\WheelCarBottomSpace}{1.4mm}

% =====================================================================
% -- Posición vertical --
% =====================================================================
\newcommand{\WheelSideLift}{-3mm}
\newcommand{\WheelCarDrop}{-3mm}

% -- Layout horizontal (valores originales del carro) --
\newcommand{\WheelSideColWidth}{0.39\linewidth}
\newcommand{\WheelCarColWidth}{0.16\linewidth}
\newcommand{\WheelGapLeftToCar}{0.03\linewidth}
\newcommand{\WheelGapCarToRight}{0.03\linewidth}
\newcommand{\WheelCardRaise}{\WheelSideLift}
\newcommand{\WheelCarRaise}{\WheelCarDrop}

% =====================================================================
% -- Comandos de celda --
% =====================================================================
\newcommand{\WheelMetricCard}[2]{%
  \begin{tcolorbox}[enhanced,nobeforeafter,
    colback=warningyellow,colframe=warningyellow,
    boxrule=0pt,arc=1.2mm,
    width=\linewidth,height=\WheelMetricTitleH,
    boxsep=0mm,
    left=0.8mm,right=0.8mm,top=0mm,bottom=0mm,
    valign=center,halign=center
  ]
    {\fontsize{\WheelMetricTitleSize}{\WheelMetricTitleLead}\selectfont\bfseries\centering #1}
  \end{tcolorbox}%
  \vspace{\WheelMetricGap}
  \begin{tcolorbox}[enhanced,nobeforeafter,
    colback=gray!15,colframe=gray!15,
    boxrule=0pt,arc=1.2mm,
    width=\linewidth,height=\WheelMetricValueH,
    boxsep=0mm,
    left=0.8mm,right=0.8mm,top=0mm,bottom=0mm,
    valign=center,halign=center
  ]
    {\fontsize{\WheelMetricValueSize}{\WheelMetricValueLead}\selectfont\centering #2}
  \end{tcolorbox}%
}

\newcommand{\WheelBrakeCellYellow}[1]{%
  \begin{tcolorbox}[enhanced,nobeforeafter,
    colback=\WheelBrakeYellowBG,colframe=\WheelBrakeYellowBG,
    boxrule=0pt,arc=1.1mm,
    width=\linewidth,height=\WheelBrakeRowH,
    boxsep=0mm,left=0.8mm,right=0.8mm,top=0mm,bottom=0mm,
    valign=center,halign=center
  ]{{\fontsize{\WheelBrakeFontSize}{\WheelBrakeFontLead}\selectfont\bfseries #1}}\end{tcolorbox}%
}

\newcommand{\WheelBrakeCellGrayGreen}[1]{%
  \begin{tcolorbox}[enhanced,nobeforeafter,
    colback=\WheelBrakeGrayBG,colframe=\WheelBrakeGrayBG,
    boxrule=0pt,arc=1.1mm,
    width=\linewidth,height=\WheelBrakeRowH,
    boxsep=0mm,left=0.8mm,right=0.8mm,top=0mm,bottom=0mm,
    valign=center,halign=center
  ]{{\fontsize{\WheelBrakeFontSize}{\WheelBrakeFontLead}\selectfont\color{\WheelBrakeGreenText} #1}}\end{tcolorbox}%
}

\newcommand{\WheelBrakeRow}[4]{%
  \setlength{\tabcolsep}{0pt}%
  \begin{tabular}{@{}p{\WheelBrakeColA}@{\hspace{\WheelBrakeGap}}p{\WheelBrakeColB}@{\hspace{\WheelBrakeGap}}p{\WheelBrakeColC}@{\hspace{\WheelBrakeGap}}p{\WheelBrakeColD}@{}}
    \WheelBrakeCellYellow{#1} &
    \WheelBrakeCellGrayGreen{#2} &
    \WheelBrakeCellYellow{#3} &
    \WheelBrakeCellGrayGreen{#4} \\
  \end{tabular}%
}

% -- Llanta y modelo --
\newcommand{\WheelLogoFile}{Goodyear.png}
\newcommand{\WheelModelName}{WRANGLER ALL TERRAIN ADVENTURE W/KEVLAR 110T}

% =====================================================================
% -- Bloque de rueda
% =====================================================================
\newcommand{\SimpleWheelBlock}[8]{%
  {\centering\bfseries\kodemono #1\par}
  \vspace{0.8mm}

  \begin{tcolorbox}[enhanced,nobeforeafter,
    width=\linewidth,
    colback=white,
    colframe=\WheelOuterBorderColor,
    boxrule=\WheelOuterBorderWidth,
    arc=\WheelOuterBorderRadius,
    boxsep=0mm,
    left=0mm,right=0mm,top=0mm,bottom=0mm
  ]
    % -- Encabezado: logo | modelo --
    \setlength{\tabcolsep}{0pt}
    \begin{tabular}{@{}p{\WheelLogoColRatio\linewidth}p{\WheelModelColRatio\linewidth}@{}}
      % Celda logo: sin padding para que la imagen llene toda la celda
      \begin{tcolorbox}[enhanced,nobeforeafter,
        width=\linewidth,height=\WheelHeaderHeight,
        colback=\WheelHeaderBrandBG,colframe=\WheelHeaderBorderColor,
        boxrule=\WheelHeaderBorderWidth,arc=0mm,
        boxsep=0mm,left=0mm,right=0mm,top=0mm,bottom=0mm,
        valign=center,halign=center
      ]
        \includegraphics[width=\WheelLogoMaxWidth,height=\WheelLogoMaxHeight,keepaspectratio]{#2}
      \end{tcolorbox}
      &
      \begin{tcolorbox}[enhanced,nobeforeafter,
        width=\linewidth,height=\WheelHeaderHeight,
        colback=\WheelHeaderModelBG,colframe=\WheelHeaderBorderColor,
        boxrule=\WheelHeaderBorderWidth,arc=0mm,
        boxsep=0mm,left=1.0mm,right=1.0mm,top=0mm,bottom=0mm,
        valign=center,halign=flush left
      ]
        {\fontsize{\WheelInfoFontSize}{\WheelInfoFontLead}\selectfont #3}
      \end{tcolorbox}
      \\
    \end{tabular}

    \vspace{0.8mm}

    % -- Datos de la llanta (estilo tarjeta con título/valor) --
    \setlength{\tabcolsep}{0pt}
    \begin{tabular}{@{\hspace{\WheelMetricSidePad}}p{\WheelMetricFiveColW}@{\hspace{\WheelMetricColGap}}p{\WheelMetricFiveColW}@{\hspace{\WheelMetricColGap}}p{\WheelMetricFiveColW}@{\hspace{\WheelMetricColGap}}p{\WheelMetricFiveColW}@{\hspace{\WheelMetricColGap}}p{\WheelMetricDepthColW}@{\hspace{\WheelMetricSidePad}}}
      \WheelMetricCard{MEDIDA}{#4} &
      \WheelMetricCard{PSI}{#5} &
      \WheelMetricCard{\shortstack[c]{ÍNDICE DE\\VELOCIDAD}}{210} &
      \WheelMetricCard{\shortstack[c]{ÍNDICE DE\\CARGA}}{210} &
      \WheelMetricCard{PROFUNDIDAD}{#6} \\
    \end{tabular}

    \vspace{\WheelMetricRowGap}

    \begin{tcolorbox}[enhanced,nobeforeafter,
      colback=\WheelBrakeHeaderBG,colframe=\WheelBrakeHeaderBG,
      boxrule=0pt,arc=0mm,
      width=\linewidth,height=\WheelBrakeHeaderH,
      boxsep=0mm,left=0mm,right=0mm,top=0mm,bottom=0mm,
      valign=center,halign=center
    ]
      {\fontsize{\WheelBrakeHeaderTextSize}{\WheelBrakeHeaderTextLead}\selectfont\bfseries FRENOS, DISCOS Y TAMBORES}
    \end{tcolorbox}

    \vspace{\WheelMetricRowGap}

    \noindent\makebox[\linewidth][c]{\WheelBrakeRow{Balatas/Tambores}{BUENA (>9MM)}{Desgaste}{#8}}%

    \vspace{\WheelMetricRowGap}

    \noindent\makebox[\linewidth][c]{\WheelBrakeRow{Discos}{BUENA (>9MM)}{Desgaste}{#7}}%
  \end{tcolorbox}
}

% =====================================================================
% -- Layout 3 columnas: tarjetas | carro | tarjetas
% =====================================================================
\noindent\makebox[\textwidth][c]{%
\begin{minipage}{\contentwidth}

  \raisebox{\WheelCardRaise}{%
  \begin{minipage}[t]{\WheelSideColWidth}
    \SimpleWheelBlock{Delantera Derecha}{\WheelLogoFile}{\WheelModelName}{315/35/R21}{25}{3 mm}{9 mm}{8 mm}
    \vspace{2.2mm}
    \SimpleWheelBlock{Trasera Derecha}{\WheelLogoFile}{\WheelModelName}{315/35/R21}{25}{3 mm}{9 mm}{8 mm}
  \end{minipage}%
  }%
  \hspace*{\WheelGapLeftToCar}%
  \raisebox{\WheelCarRaise}{%
  \begin{minipage}[t]{\WheelCarColWidth}
    \vspace{\WheelCarTopSpace}
    {\noindent\makebox[\linewidth][c]{\hspace*{\WheelCarShiftX}\includegraphics[width=\WheelCarWidth,keepaspectratio]{\WheelCarFile}}\par}
    \vspace{\WheelCarBottomSpace}
  \end{minipage}%
  }%
  \hspace*{\WheelGapCarToRight}%
  \raisebox{\WheelCardRaise}{%
  \begin{minipage}[t]{\WheelSideColWidth}
    \SimpleWheelBlock{Delantera Izquierda}{\WheelLogoFile}{\WheelModelName}{315/35/R21}{25}{3 mm}{9 mm}{8 mm}
    \vspace{2.2mm}
    \SimpleWheelBlock{Trasera Izquierda}{\WheelLogoFile}{\WheelModelName}{315/35/R21}{25}{3 mm}{9 mm}{8 mm}
  \end{minipage}%
  }

\end{minipage}%
}

% Texto opcional dentro de observaciones (vacío por defecto)
\newcommand{\WheelObservacionesTexto}{}
\newcommand{\WheelObservacionesPlaceholder}{\color{gray!55}Escribe aquí observaciones del servicio.}

\vspace{-4.04mm}

% -- Observaciones (alineado al mismo margen izquierdo/derecho del contenido) --
\noindent\makebox[\textwidth][c]{%
\begin{minipage}{\contentwidth}
  {\fontsize{8}{12}\selectfont Observaciones}
  \vspace{1mm}

  \begin{tcolorbox}[
    enhanced,
    nobeforeafter,
    width=\linewidth,
    height=18mm,
    colback=gray!15,
    colframe=gray!15,
    boxrule=0pt,
    arc=2.5mm,
    left=3.2mm,
    right=3.2mm,
    top=2.3mm,
    bottom=2.3mm,
    valign=top
  ]
    {\fontsize{7}{11}\selectfont
      \ifx\WheelObservacionesTexto\empty
        \WheelObservacionesPlaceholder
      \else
        \WheelObservacionesTexto
      \fi
    }
  \end{tcolorbox}
\end{minipage}%
}

\vspace{5.2mm}

% -- Título: Sistema de suspensión --
\noindent\makebox[\textwidth][c]{%
  \begin{tcolorbox}[
    colback=gray!25,
    colframe=gray!25,
    width=\contentwidth,
    arc=1mm,
    boxrule=0pt,
    left=\padlr,
    right=\padlr,
    top=1mm,
    bottom=1mm
  ]
    \centering
    {\fontsize{15}{18}\selectfont \bfseries Sistema de suspensión}
  \end{tcolorbox}%
}

\vspace{1.8mm}

% -- Selector 4 estados para suspensión (estilo similar a Anticongelante) --
% #1 = estado seleccionado (0..3)
% #2 = título del bloque
% #3 = observación
\newcommand{\SuspensionSelector}[3]{%
  \begin{minipage}{\linewidth}
    {\fontsize{7}{11}\selectfont #2}
    \vspace{0.9mm}

    \begin{tikzpicture}
      \pgfmathsetlengthmacro{\W}{\linewidth}
      \pgfmathsetlengthmacro{\H}{16mm}
      \pgfmathsetlengthmacro{\Box}{\W/4}

      \pgfmathsetlengthmacro{\R}{2.9mm}
      \pgfmathsetlengthmacro{\yCircle}{10.2mm}
      \pgfmathsetlengthmacro{\LabelH}{6.8mm}
      \pgfmathsetlengthmacro{\yLbl}{3.6mm}
      \pgfmathsetlengthmacro{\CardRad}{2mm}

      \begin{scope}
        \clip[rounded corners=\CardRad] (0,0) rectangle (\W,\H);

        \newcommand{\sitem}[4]{%
          \ifnum##2=1
            \path[draw=none, fill=gray!15] (##1,0) rectangle (##1+\Box,\H);
          \fi

          \fill[##3] (##1+0.5*\Box,\yCircle) circle (\R);

          \ifnum##2=1
            \node[
              align=center,
              text=black,
              font=\fontsize{7.6}{8.8}\selectfont,
              text width=\Box,
              minimum height=\LabelH,
              anchor=center
            ] at (##1+0.5*\Box,\yLbl) {##4};
          \else
            \node[
              align=center,
              text=gray!55,
              font=\fontsize{7.6}{8.8}\selectfont,
              text width=\Box,
              minimum height=\LabelH,
              anchor=center
            ] at (##1+0.5*\Box,\yLbl) {##4};
          \fi
        }

        \pgfmathsetlengthmacro{\xA}{0*\Box}
        \pgfmathsetlengthmacro{\xB}{1*\Box}
        \pgfmathsetlengthmacro{\xC}{2*\Box}
        \pgfmathsetlengthmacro{\xD}{3*\Box}

        \sitem{\xA}{\ifnum#1=0 1\else 0\fi}{blue!45}{N/A}
        \sitem{\xB}{\ifnum#1=1 1\else 0\fi}{green!60!black}{Buena}
        \sitem{\xC}{\ifnum#1=2 1\else 0\fi}{warningyellow}{Regular}
        \sitem{\xD}{\ifnum#1=3 1\else 0\fi}{red!85!black}{Mala}
      \end{scope}

      \path[draw=black!15, fill=none, rounded corners=\CardRad, line width=0.5pt]
        (0,0) rectangle (\W,\H);
    \end{tikzpicture}

    \vspace{1mm}
    {\fontsize{7.2}{11}\selectfont Observaciones}
    \vspace{0.8mm}

    \begin{tcolorbox}[
      enhanced,
      nobeforeafter,
      width=\linewidth,
      height=8.8mm,
      colback=gray!15,
      colframe=gray!15,
      boxrule=0pt,
      arc=1.8mm,
      left=1.8mm,
      right=1.8mm,
      top=1.2mm,
      bottom=1.2mm,
      valign=center
    ]
      {\fontsize{7.2}{8.5}\selectfont #3}
    \end{tcolorbox}
  \end{minipage}%
}

% -- Grid de suspensión (3 columnas, alineado al ancho del título) --
\noindent\makebox[\textwidth][c]{%
\begin{minipage}{\contentwidth}
  \begin{minipage}[t]{0.31\linewidth}
    \SuspensionSelector{0}{Baleros de rueda / Delanteros y Traseros}{CON MASA}
  \end{minipage}\hfill
  \begin{minipage}[t]{0.31\linewidth}
    \SuspensionSelector{3}{Cremallera / Caja de Dirección}{FUGA DE ACEITE}
  \end{minipage}\hfill
  \begin{minipage}[t]{0.31\linewidth}
    \SuspensionSelector{3}{Horquillas, Rótulas o articulaciones}{CAMBIO SUPERIOR E INFERIOR}
  \end{minipage}
\end{minipage}%
}

% Configuración del bloque adicional solicitado en suspensión
\newcommand{\SuspensionExtraTitulo}{Barra estabilizadora y sus componentes}
\newcommand{\SuspensionExtraEstado}{0}
\newcommand{\SuspensionExtraObservacionesTexto}{CON MASA, FUGA DE ACEITE, CAMBIO SUPERIOR E INFERIOR.}

% Selector compacto de estado (N/A, Buena, Regular, Mala)
\newcommand{\SuspensionStatusSelector}[1]{%
  \begin{tikzpicture}
    \pgfmathsetlengthmacro{\W}{\linewidth}
    \pgfmathsetlengthmacro{\H}{16mm}
    \pgfmathsetlengthmacro{\Box}{\W/4}

    \pgfmathsetlengthmacro{\R}{2.9mm}
    \pgfmathsetlengthmacro{\yCircle}{10.2mm}
    \pgfmathsetlengthmacro{\LabelH}{6.8mm}
    \pgfmathsetlengthmacro{\yLbl}{3.6mm}
    \pgfmathsetlengthmacro{\CardRad}{2mm}

    \begin{scope}
      \clip[rounded corners=\CardRad] (0,0) rectangle (\W,\H);

      \newcommand{\ssitem}[4]{%
        \ifnum##2=1
          \path[draw=none, fill=gray!15] (##1,0) rectangle (##1+\Box,\H);
        \fi

        \fill[##3] (##1+0.5*\Box,\yCircle) circle (\R);

        \ifnum##2=1
          \node[
            align=center,
            text=black,
            font=\fontsize{7.6}{8.8}\selectfont,
            text width=\Box,
            minimum height=\LabelH,
            anchor=center
          ] at (##1+0.5*\Box,\yLbl) {##4};
        \else
          \node[
            align=center,
            text=gray!55,
            font=\fontsize{7.6}{8.8}\selectfont,
            text width=\Box,
            minimum height=\LabelH,
            anchor=center
          ] at (##1+0.5*\Box,\yLbl) {##4};
        \fi
      }

      \pgfmathsetlengthmacro{\xA}{0*\Box}
      \pgfmathsetlengthmacro{\xB}{1*\Box}
      \pgfmathsetlengthmacro{\xC}{2*\Box}
      \pgfmathsetlengthmacro{\xD}{3*\Box}

      \ssitem{\xA}{\ifnum#1=0 1\else 0\fi}{blue!45}{N/A}
      \ssitem{\xB}{\ifnum#1=1 1\else 0\fi}{green!60!black}{Buena}
      \ssitem{\xC}{\ifnum#1=2 1\else 0\fi}{warningyellow}{Regular}
      \ssitem{\xD}{\ifnum#1=3 1\else 0\fi}{red!85!black}{Mala}
    \end{scope}

    \path[draw=black!15, fill=none, rounded corners=\CardRad, line width=0.5pt]
      (0,0) rectangle (\W,\H);
  \end{tikzpicture}%
}

\vspace{1.6mm}

% -- Fila adicional: selector + observaciones --
\noindent\makebox[\textwidth][c]{%
\begin{minipage}{\contentwidth}
  \begin{minipage}[t]{0.30\linewidth}
    {\fontsize{7}{11}\selectfont \SuspensionExtraTitulo}
    \vspace{0.9mm}

    \SuspensionStatusSelector{\SuspensionExtraEstado}
  \end{minipage}\hfill
  \begin{minipage}[t]{0.68\linewidth}
    {\fontsize{7.2}{11}\selectfont Observaciones}
    \vspace{0.8mm}

    \begin{tcolorbox}[
      enhanced,
      nobeforeafter,
      width=\linewidth,
      height=16mm,
      colback=gray!15,
      colframe=gray!15,
      boxrule=0pt,
      arc=1.8mm,
      left=1.8mm,
      right=1.8mm,
      top=1.2mm,
      bottom=1.2mm,
      valign=top
    ]
      {\fontsize{7.2}{8.5}\selectfont \SuspensionExtraObservacionesTexto}
    \end{tcolorbox}
  \end{minipage}
\end{minipage}%
}
