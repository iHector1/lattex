%% page3.tex
%% (SOLO CONTENIDO — NO PONGAS \documentclass NI \usepackage AQUÍ)

% -- Título: Amortiguadores y Bases --
\noindent\makebox[\textwidth][c]{%
  \begin{tcolorbox}[
    colback=gray!25,
    colframe=gray!25,
    width=\contentwidth,
    arc=1mm,
    boxrule=0pt,
    left=\padlr,
    right=\padlr,
    top=1mm,
    bottom=1mm
  ]
    \centering
    {\fontsize{15}{18}\selectfont \bfseries Amortiguadores y Bases}
  \end{tcolorbox}%
}

\vspace{1.6mm}

\noindent\makebox[\textwidth][c]{%
\begin{minipage}{\contentwidth}
  {\kodemono\fontseries{sb}\fontsize{13}{14}\selectfont Inspección de Amortiguadores y Bases Delanteras}
\end{minipage}%
}

\vspace{2.5mm}

\noindent\makebox[\textwidth][c]{%
\begin{minipage}{\contentwidth}
\centering
\begin{tikzpicture}[x=1mm,y=1mm]
  % Dimensiones
  \def\LabelW{26}
  \def\CellW{20}
  \def\GapCol{3}
  \def\Rows{4}
  \def\RowH{7.5}
  \def\ColGroupGap{10}
  \def\GroupW{66}
  \def\HeadH{7}
  \def\BandH{4}

  % Posiciones X
  \pgfmathsetmacro{\xLabel}{0}
  \pgfmathsetmacro{\xA}{\LabelW + 4}
  \pgfmathsetmacro{\xB}{\xA + \GroupW + \ColGroupGap}

  % Y base
  \pgfmathsetmacro{\yTop}{0}

  % Encabezados grupos
  \fill[warningyellow] (\xA,\yTop) rectangle ++(\GroupW,\HeadH);
  \fill[warningyellow] (\xB,\yTop) rectangle ++(\GroupW,\HeadH);
  \node[font=\bfseries\fontsize{10.5}{11}\selectfont] at (\xA+\GroupW/2,\yTop+\HeadH/2) {AMORTIGUADORES};
  \node[font=\bfseries\fontsize{10.5}{11}\selectfont] at (\xB+\GroupW/2,\yTop+\HeadH/2) {BASES};

  % Sub-bandas de estado (buena/regular/mala)
  \foreach \x in {\xA,\xB} {
    \fill[green!65!black] (\x,\yTop-\BandH) rectangle ++(\CellW,\BandH);
    \fill[warningyellow] (\x+\CellW+\GapCol,\yTop-\BandH) rectangle ++(\CellW,\BandH);
    \fill[red!85!black] (\x+2*\CellW+2*\GapCol,\yTop-\BandH) rectangle ++(\CellW,\BandH);

    \node[font=\bfseries\fontsize{6.8}{7}\selectfont,text=white] at (\x+\CellW/2,\yTop-\BandH/2) {BUENA};
    \node[font=\bfseries\fontsize{6.8}{7}\selectfont,text=white] at (\x+\CellW+\GapCol+\CellW/2,\yTop-\BandH/2) {REGULAR};
    \node[font=\bfseries\fontsize{6.8}{7}\selectfont,text=white] at (\x+2*\CellW+2*\GapCol+\CellW/2,\yTop-\BandH/2) {MALA};
  }

  % Filas
  \foreach \i/\rowname in {0/DELANTERA D.,1/DELANTERA I.,2/TRASERA D.,3/TRASERA I.} {
    \pgfmathsetmacro{\yRow}{\yTop-\HeadH-\BandH-2 - \i*\RowH}

    % Etiquetas izquierdas
    \fill[warningyellow] (\xLabel,\yRow-5.8) rectangle ++(\LabelW,5.8);
    \node[anchor=west,font=\bfseries\fontsize{9.3}{10}\selectfont] at (1.2,\yRow-2.9) {\rowname};

    % Celdas Amortiguadores
    \foreach \j in {0,1,2} {
      \fill[gray!15,rounded corners=1.2mm] (\xA+\j*(\CellW+\GapCol),\yRow-5.8) rectangle ++(\CellW,5.8);
      \fill[gray!15,rounded corners=1.2mm] (\xB+\j*(\CellW+\GapCol),\yRow-5.8) rectangle ++(\CellW,5.8);
    }
  }

  % Puntos de ejemplo en AMORTIGUADORES (como referencia)
  % fila 1 buena
  \fill[green!65!black] (\xA+\CellW/2,\yTop-\HeadH-\BandH-2-2.9) circle (2.1);
  % fila 2 regular
  \fill[warningyellow] (\xA+\CellW+\GapCol+\CellW/2,\yTop-\HeadH-\BandH-2-\RowH-2.9) circle (2.1);
  % fila 3 mala
  \fill[red!85!black] (\xA+2*\CellW+2*\GapCol+\CellW/2,\yTop-\HeadH-\BandH-2-2*\RowH-2.9) circle (2.1);
  % fila 4 buena
  \fill[green!65!black] (\xA+\CellW/2,\yTop-\HeadH-\BandH-2-3*\RowH-2.9) circle (2.1);

\end{tikzpicture}
\end{minipage}%
}
