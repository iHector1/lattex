%% main.tex
%% vergueado no, vergas si --Mike 2026

\documentclass{article}

% =========================
% PAQUETES
% =========================
\usepackage{geometry}
\usepackage{graphicx}
\usepackage{fontspec}
\usepackage[most]{tcolorbox}
\usepackage{xcolor} % <-- xcolor ANTES de definecolor
\usepackage{tabularx}
\usepackage{tikz}
\usetikzlibrary{calc}
%% configuraciones/rutas.tex
%% Rutas centralizadas del proyecto

\graphicspath{{./imagenes/}}



% =========================
% COLORES
% =========================
\definecolor{warningyellow}{HTML}{FCC01A}
\definecolor{trackgray}{HTML}{E6E6E6}
\definecolor{oiltrack}{HTML}{E6E6E6}

% =========================
% CONFIGURACIÓN DE PÁGINA
% =========================
\geometry{
  margin=0cm,
  top=0cm,
  bottom=2cm,
  left=0cm,
  right=0cm
}

\pagestyle{empty}
\parindent=0pt
\parskip=0pt

% =========================
% FUENTES
% =========================
\setmainfont{Montserrat}

% Kode Mono Variable (archivo local en la carpeta)
\newfontfamily\kodemono[
  Path=./configuraciones/,
  Extension=.ttf,
  UprightFont=*,
  FontFace={l}{n}{Font=*, RawFeature={+wght=300}},
  FontFace={m}{n}{Font=*, RawFeature={+wght=400}},
  FontFace={sb}{n}{Font=*, RawFeature={+wght=600}},
  FontFace={b}{n}{Font=*, RawFeature={+wght=700}}
]{KodeMono-VariableFont_wght}

% =========================
% MEDIDAS
% =========================
\newcommand{\contentwidth}{0.90\textwidth}

\newlength{\padlr}
\setlength{\padlr}{0mm}

\newlength{\innerwidth}
\setlength{\innerwidth}{\dimexpr \contentwidth - 2\padlr \relax}

% =========================
% ESTILOS
% =========================
\newtcolorbox{inputbox}{
  colback=white,
  colframe=black!15,
  arc=1mm,
  boxrule=0.4pt,
  left=2mm,
  right=2mm,
  top=1mm,
  bottom=1mm,
  height=7mm,
  valign=center
}

\newtcolorbox{commentbox}{
  colback=gray!10,
  colframe=black!15,
  arc=1.5mm,
  boxrule=0.5pt,
  left=4mm,
  right=4mm,
  top=3mm,
  bottom=3mm,
  height=18mm,      % ajustable
  valign=top
}

\newcommand{\fieldlabel}[1]{%
  {\fontsize{8.5}{10}\selectfont #1}\vspace{1mm}%
}

\newcommand{\uilabel}{\fontsize{8.5}{10}\selectfont\bfseries}
\newcommand{\uilabelgray}{\fontsize{8.5}{10}\selectfont\bfseries\color{gray!60}}

% =========================
% CHIPS (ETIQUETAS)
% =========================
\newcommand{\chip}[1]{%
  \tikz[baseline]
    \node[
      draw=none,
      fill=gray!15,
      rounded corners=2mm,
      inner xsep=3mm,
      inner ysep=1mm
    ]{\fontsize{9}{11}\selectfont #1};%
}

% =========================
% ICONOS (CÍRCULOS + PNG)
% =========================
\newcommand{\warniconsize}{0.72cm} % PNG
\newcommand{\warnradius}{0.58}     % radio círculo
\newcommand{\warngap}{0.45}        % separación horizontal
\newcommand{\warnrowgap}{1.45}     % separación vertical

\newcommand{\warnicon}[4]{%
  \ifnum#4=1
    \fill[warningyellow] (#1,#2) circle (\warnradius);
  \else
    \fill[gray!15] (#1,#2) circle (\warnradius);
  \fi
  \node at (#1,#2) {\includegraphics[width=\warniconsize]{#3}};
}

\newcommand{\warniconij}[4]{%
  \pgfmathsetmacro{\step}{2*\warnradius + \warngap}%
  \pgfmathsetmacro{\xx}{(#1)*\step}%
  \pgfmathsetmacro{\yy}{-(#2)*\warnrowgap}%
  \warnicon{\xx}{\yy}{#3}{#4}%
}

% =========================
% BARRA NIVEL DE GASOLINA
% =========================
\newcommand{\gasfont}{\fontsize{9}{11}\selectfont\bfseries}

% #1 = nivel seleccionado: 0..4
\newcommand{\gasolinebar}[1]{%
  \noindent\makebox[\innerwidth][c]{%
    \begin{tikzpicture}

      \pgfmathsetlengthmacro{\W}{0.90\innerwidth}
      \pgfmathsetlengthmacro{\H}{3.2mm}
      \pgfmathsetlengthmacro{\R}{1.6mm}
      \pgfmathsetlengthmacro{\KnobR}{2.55mm}

      \pgfmathsetlengthmacro{\yBar}{0mm}
      \pgfmathsetlengthmacro{\yKnob}{-6.0mm}
      \pgfmathsetlengthmacro{\yLabel}{-10.5mm}
      \pgfmathsetlengthmacro{\yEF}{0.2mm}

      \useasboundingbox (0,\yLabel-4mm) rectangle (\W,\yBar+4mm);

      \coordinate (L) at (0,0);
      \coordinate (Rr) at (\W,0);

      % PUNTOS internos SOLO para bolitas (no para el fill)
      \coordinate (Cleft)  at ($(L)!0.01!(Rr)$);
      \coordinate (Cright) at (Rr);

      \coordinate (P0) at (Cleft);
      \coordinate (P1) at ($(Cleft)!0.25!(Cright)$);
      \coordinate (P2) at ($(Cleft)!0.50!(Cright)$);
      \coordinate (P3) at ($(Cleft)!0.75!(Cright)$);
      \coordinate (P4) at (Cright);

      % Track gris (real, de L a R)
      \path[draw=none, fill=trackgray, rounded corners=\R]
        ($(L)+(0,\yBar-\H/2)$) rectangle ($(Rr)+(0,\yBar+\H/2)$);

      % Punto final del fill:
      %   - 0..3 llega a P0..P3 (alineado con bolitas)
      %   - 4 (LLENO) llega a R real (hasta el borde)
      \ifcase#1
        \coordinate (PF) at (P0);
      \or
        \coordinate (PF) at (P1);
      \or
        \coordinate (PF) at (P2);
      \or
        \coordinate (PF) at (P3);
      \or
        \coordinate (PF) at (Rr);
      \fi

      % Fill amarillo (SIEMPRE empieza en L real)
      \path[draw=none, fill=warningyellow, rounded corners=\R]
        ($(L)+(0,\yBar-\H/2)$) rectangle ($(PF)+(0,\yBar+\H/2)$);

      % E y F
      \node[anchor=east] at ($(L)+(-0.5mm,\yEF)$) {{\gasfont E}};
      \node[anchor=west] at ($(Rr)+( 0.5mm,\yEF)$) {{\gasfont F}};

      % knobs
      \newcommand{\knob}[3]{%
        \ifnum##2=1
          \fill[warningyellow] ($ (##1) + (0,\yKnob) $) circle (\KnobR);
        \else
          \fill[trackgray] ($ (##1) + (0,\yKnob) $) circle (\KnobR);
        \fi
        \ifnum##2=1
          \node[anchor=north, text=black] at ($ (##1) + (0,\yLabel) $) {{\gasfont ##3}};
        \else
          \node[anchor=north, text=gray!55] at ($ (##1) + (0,\yLabel) $) {{\gasfont ##3}};
        \fi
      }

      \knob{P0}{\ifnum#1=0 1\else 0\fi}{BAJO}
      \knob{P1}{\ifnum#1=1 1\else 0\fi}{1/4}
      \knob{P2}{\ifnum#1=2 1\else 0\fi}{MEDIO}
      \knob{P3}{\ifnum#1=3 1\else 0\fi}{3/4}
      \knob{P4}{\ifnum#1=4 1\else 0\fi}{LLENO}

    \end{tikzpicture}%
  }%
}

% =========================
% LÍQUIDOS: NIVEL DE ACEITE (UI - texto gris claro)
% =========================
% #1 = estado: 0=DEBAJO, 1=A NIVEL, 2=ARRIBA
\newcommand{\oillevelbar}[1]{%
  \begin{tikzpicture}

    \pgfmathsetlengthmacro{\W}{0.97\linewidth}
    \pgfmathsetlengthmacro{\H}{3.0mm}
    \pgfmathsetlengthmacro{\R}{1.5mm}
    \pgfmathsetlengthmacro{\DotR}{1.7mm}

    \pgfmathsetlengthmacro{\yBar}{0mm}
    \pgfmathsetlengthmacro{\yDot}{-4.0mm}
    \pgfmathsetlengthmacro{\yLbl}{-6.2mm}

    \useasboundingbox (0,\yLbl-3mm) rectangle (\W,\yBar+3mm);

    \coordinate (L) at (0,0);
    \coordinate (Rr) at (\W,0);

    % margen interno SOLO para bolitas/labels
    \pgfmathsetlengthmacro{\xPad}{3mm}
    \coordinate (P0) at ($(L)+(\xPad,0)$);
    \coordinate (P2) at ($(Rr)-(\xPad,0)$);
    \coordinate (P1) at ($(P0)!0.50!(P2)$);

    % Track (real, de L a R)
    \path[draw=none, fill=oiltrack, rounded corners=\R]
      ($(L)+(0,\yBar-\H/2)$) rectangle ($(Rr)+(0,\yBar+\H/2)$);

    % --- PFfill = fin del relleno | PFdot = dónde va la bolita amarilla ---
    \ifcase#1
      \coordinate (PFfill) at (P0);
      \coordinate (PFdot)  at (P0);
    \or
      \coordinate (PFfill) at (P1);
      \coordinate (PFdot)  at (P1);
    \or
      \coordinate (PFfill) at (Rr); % barra hasta el borde
      \coordinate (PFdot)  at (P2); % bolita se queda en el punto
    \fi

    % Fill amarillo (SIEMPRE desde L real)
    \path[draw=none, fill=warningyellow, rounded corners=\R]
      ($(L)+(0,\yBar-\H/2)$) rectangle ($(PFfill)+(0,\yBar+\H/2)$);

    % 3 puntos grises + punto activo
    \fill[oiltrack]      ($ (P0) + (0,\yDot) $) circle (\DotR);
    \fill[oiltrack]      ($ (P1) + (0,\yDot) $) circle (\DotR);
    \fill[oiltrack]      ($ (P2) + (0,\yDot) $) circle (\DotR);
    \fill[warningyellow] ($ (PFdot) + (0,\yDot) $) circle (\DotR);

    % Labels (negro si seleccionado, gris si no)
    \ifnum#1=0
      \node[anchor=north, text=black, align=center, xshift=10mm] at ($ (P0) + (0,\yLbl) $)
        {\fontsize{6.5}{8}\selectfont\bfseries DEBAJO DEL NIVEL};
    \else
      \node[anchor=north, text=gray!55, align=center, xshift=10mm] at ($ (P0) + (0,\yLbl) $)
        {\fontsize{6.5}{8}\selectfont\bfseries DEBAJO DEL NIVEL};
    \fi

    \ifnum#1=1
      \node[anchor=north, text=black, align=center] at ($ (P1) + (0,\yLbl) $)
        {\fontsize{6.5}{8}\selectfont\bfseries A NIVEL};
    \else
      \node[anchor=north, text=gray!55, align=center] at ($ (P1) + (0,\yLbl) $)
        {\fontsize{6.5}{8}\selectfont\bfseries A NIVEL};
    \fi

    \ifnum#1=2
      \node[anchor=north, text=black, align=center, xshift=-10mm] at ($ (P2) + (0,\yLbl) $)
        {\fontsize{6.5}{8}\selectfont\bfseries ARRIBA DEL NIVEL};
    \else
      \node[anchor=north, text=gray!55, align=center, xshift=-10mm] at ($ (P2) + (0,\yLbl) $)
        {\fontsize{6.5}{8}\selectfont\bfseries ARRIBA DEL NIVEL};
    \fi

  \end{tikzpicture}%
}

% =========================
% LÍQUIDOS: COLOR DE ACEITE (RESPONSIVO + ESTÉTICO)
% =========================
% #1 = 0 Limpio, 1 Medio, 2 Quemado
\newcommand{\oilcolorselector}[1]{%
  \begin{tikzpicture}
    % Ancho disponible = ancho de la columna
    \pgfmathsetlengthmacro{\W}{\linewidth}

    % Gap proporcional (se ve bien en tamaños chicos y grandes)
    \pgfmathsetlengthmacro{\Gap}{0.06*\W}

    % Caja auto-escalada para que quepan 3 items + 2 gaps
    \pgfmathsetlengthmacro{\Box}{(\W - 2*\Gap)/3}

    % Radio del círculo proporcional a la caja
    \pgfmathsetlengthmacro{\R}{0.30*\Box}

    % Alturas (proporcionales)
    \pgfmathsetlengthmacro{\yCircle}{0.62*\Box}
    \pgfmathsetlengthmacro{\yLbl}{0.18*\Box}

    % Fondo (para que TikZ reserve espacio correcto)
    \useasboundingbox (0,-0.12*\Box) rectangle (\W,\Box);

    \newcommand{\citem}[4]{%
      % ##1 x, ##2 selected 0/1, ##3 fill, ##4 text

      % Fondo seleccionado (suave)
      \ifnum##2=1
        \path[draw=none, fill=gray!15, rounded corners=2mm]
          (##1,-0.12*\Box) rectangle (##1+\Box,\Box);
      \fi

      % Círculo
      \fill[##3] (##1+0.5*\Box,\yCircle) circle (\R);

      % Texto (negro si seleccionado, gris si no)
      \ifnum##2=1
        \node[anchor=north, text=black] at (##1+0.5*\Box,\yLbl)
          {\fontsize{8.5}{10}\selectfont\bfseries ##4};
      \else
        \node[anchor=north, text=gray!55] at (##1+0.5*\Box,\yLbl)
          {\fontsize{8.5}{10}\selectfont\bfseries ##4};
      \fi
    }

    \citem{0}{\ifnum#1=0 1\else 0\fi}{warningyellow}{Limpio}
    \citem{\Box+\Gap}{\ifnum#1=1 1\else 0\fi}{brown!70!black}{Medio}
    \citem{2*\Box+2*\Gap}{\ifnum#1=2 1\else 0\fi}{black!80}{Quemado}
  \end{tikzpicture}%
}

% =========================
% SELECTOR 4 OPCIONES (compacto estilo referencia + bordes llenos)
% =========================
% #1 = seleccionado (0..3)
% #2 = titulo
\newcommand{\fluidselectorfour}[2]{%
  \begin{minipage}{\linewidth}
    {\fontsize{9}{11}\selectfont #2}
    \vspace{1mm}

    \begin{tikzpicture}
      \pgfmathsetlengthmacro{\W}{\linewidth}
      \pgfmathsetlengthmacro{\H}{16mm}
      \pgfmathsetlengthmacro{\Box}{\W/4}

      \pgfmathsetlengthmacro{\R}{2.9mm}
      \pgfmathsetlengthmacro{\yCircle}{10.2mm}
      \pgfmathsetlengthmacro{\LabelH}{6.8mm}
      \pgfmathsetlengthmacro{\yLbl}{3.6mm}
      \pgfmathsetlengthmacro{\CardRad}{2mm}

      \begin{scope}
        \clip[rounded corners=\CardRad] (0,0) rectangle (\W,\H);

        \newcommand{\fitem}[5]{%
          \ifnum##2=1
            \path[draw=none, fill=gray!15] (##1,0) rectangle (##1+\Box,\H);
          \fi

          \fill[##3] (##1+0.5*\Box,\yCircle) circle (\R);

          \ifnum##2=1
            \node[
              align=center,
              text=black,
              font=\fontsize{6.5}{7.5}\selectfont\bfseries,
              text width=\Box,
              minimum height=\LabelH,
              anchor=center
            ] at (##1+0.5*\Box,\yLbl) {##4\ifx##5\empty\else\\##5\fi};
          \else
            \node[
              align=center,
              text=gray!55,
              font=\fontsize{6.5}{7.5}\selectfont\bfseries,
              text width=\Box,
              minimum height=\LabelH,
              anchor=center
            ] at (##1+0.5*\Box,\yLbl) {##4\ifx##5\empty\else\\##5\fi};
          \fi
        }

        \pgfmathsetlengthmacro{\xA}{0*\Box}
        \pgfmathsetlengthmacro{\xB}{1*\Box}
        \pgfmathsetlengthmacro{\xC}{2*\Box}
        \pgfmathsetlengthmacro{\xD}{3*\Box}

        \fitem{\xA}{\ifnum#1=0 1\else 0\fi}{blue!45}{N/A}{}
        \fitem{\xB}{\ifnum#1=1 1\else 0\fi}{green!60!black}{Arriba del}{Nivel}
        \fitem{\xC}{\ifnum#1=2 1\else 0\fi}{warningyellow}{A Nivel}{}
        \fitem{\xD}{\ifnum#1=3 1\else 0\fi}{red!85!black}{Debajo del}{Nivel}

      \end{scope}

      \path[draw=black!15, fill=none, rounded corners=\CardRad, line width=0.5pt]
        (0,0) rectangle (\W,\H);
    \end{tikzpicture}
  \end{minipage}%
}

% =========================
% SELECTOR SI/NO (tipo botón)
% =========================
% #1 = 0 -> SI seleccionado, 1 -> NO seleccionado
\newcommand{\yesnoselector}[1]{%
  \begin{tikzpicture}[baseline]
    \newcommand{\btn}[3]{%
      % ##1 text, ##2 selected 0/1, ##3 x
      \ifnum##2=1
        \path[draw=none, fill=warningyellow, rounded corners=1.4mm] (##3,0) rectangle ++(10mm,6mm);
        \node[text=black] at (##3+5mm,3mm) {\fontsize{8.5}{10}\selectfont\bfseries ##1};
      \else
        \path[draw=black!25, fill=white, rounded corners=1.4mm, line width=0.5pt] (##3,0) rectangle ++(10mm,6mm);
        \node[text=black] at (##3+5mm,3mm) {\fontsize{8.5}{10}\selectfont\bfseries ##1};
      \fi
    }
    \btn{SI}{\ifnum#1=0 1\else 0\fi}{0mm}
    \btn{NO}{\ifnum#1=1 1\else 0\fi}{11.5mm}
  \end{tikzpicture}%
}
%% paginas/tirepage_components.tex
%% Componentes reorganizados para la página de Llantas y Frenos

\makeatletter
\@ifundefined{TireAndBrakesComponentsLoaded}{%
  \def\TireAndBrakesComponentsLoaded{1}%
}{%
  \endinput%
}
\makeatother

% =========================
% COLORES
% =========================
\providecolor{TFBlue}{HTML}{0B4AA2}
\providecolor{TFHeadGray}{HTML}{D9D9D9}
\providecolor{TFSoftGray}{HTML}{ECECEC}
\providecolor{TFYellow}{HTML}{F4BE16}
\providecolor{TFGreenBG}{HTML}{DFF3D7}
\providecolor{TFGreenText}{HTML}{63A83D}
\providecolor{TFText}{HTML}{1D1D1D}

% =========================
% ESCALA / MEDIDAS
% =========================
\newcommand{\TFTitleSize}{17}
\newcommand{\TFCardPad}{1.2mm}
\newcommand{\TFCardBorder}{0.6pt}
\newcommand{\TFCardW}{\linewidth}

\newcommand{\TFHeaderH}{8.2mm}
\newcommand{\TFLabelH}{5.8mm}
\newcommand{\TFValueH}{6.4mm}
\newcommand{\TFSectionH}{6.0mm}
\newcommand{\TFRowH}{5.9mm}

% =========================
% TÍTULO DE POSICIÓN
% =========================
\newcommand{\TFTitle}[1]{%
  {\centering\kodemono\fontseries{sb}\fontsize{\TFTitleSize}{20}\selectfont #1\par}
  \vspace{0.5mm}%
}

% =========================
% CABECERA DE TARJETA
% =========================
\newcommand{\TFHeader}[2]{%
  \noindent
  \begin{minipage}[c][\TFHeaderH][c]{0.35\linewidth}
    \tcbox[
      enhanced,
      colback=TFBlue,colframe=TFBlue,boxrule=0pt,arc=0mm,
      width=\linewidth,height=\TFHeaderH,
      left=0mm,right=0mm,top=0mm,bottom=0mm,
      valign=center,halign=center
    ]{\includegraphics[height=4.0mm]{#1}}%
  \end{minipage}%
  \begin{minipage}[c][\TFHeaderH][c]{0.65\linewidth}
    \tcbox[
      enhanced,
      colback=TFHeadGray,colframe=TFHeadGray,boxrule=0pt,arc=0mm,
      width=\linewidth,height=\TFHeaderH,
      left=1.5mm,right=1mm,top=0mm,bottom=0mm,
      valign=center,halign=left
    ]{{\fontsize{7.6}{8.8}\selectfont\bfseries\color{TFText} #2}}%
  \end{minipage}%
}

% =========================
% CELDA NORMAL
% =========================
\newcommand{\TFMetricCell}[2]{%
  \begin{tcolorbox}[
    enhanced,
    colback=white,colframe=white,boxrule=0pt,arc=0mm,
    left=0mm,right=0mm,top=0mm,bottom=0mm,boxsep=0mm
  ]
    \tcbox[
      enhanced,
      colback=TFYellow,colframe=TFYellow,boxrule=0pt,arc=1.3mm,
      width=\linewidth,height=\TFLabelH,
      left=0.6mm,right=0.6mm,top=0.6mm,bottom=0.6mm,
      valign=center,halign=center
    ]{{\fontsize{7.6}{8.4}\selectfont\bfseries #1}}%

    \vspace{0.5mm}%

    \tcbox[
      enhanced,
      colback=TFSoftGray,colframe=TFSoftGray,boxrule=0pt,arc=1.2mm,
      width=\linewidth,height=\TFValueH,
      left=0.6mm,right=0.6mm,top=0.6mm,bottom=0.6mm,
      valign=center,halign=center
    ]{{\fontsize{8.8}{9.8}\selectfont\bfseries #2}}%
  \end{tcolorbox}%
}

% =========================
% CELDA PSI
% =========================
\newcommand{\TFPSICell}[2]{%
  \begin{tcolorbox}[
    enhanced,
    colback=white,colframe=white,boxrule=0pt,arc=0mm,
    left=0mm,right=0mm,top=0mm,bottom=0mm,boxsep=0mm
  ]
    \tcbox[
      enhanced,
      colback=TFYellow,colframe=TFYellow,boxrule=0pt,arc=1.3mm,
      width=\linewidth,height=\TFLabelH,
      left=0.6mm,right=0.6mm,top=0.6mm,bottom=0.6mm,
      valign=center,halign=center
    ]{{\fontsize{7.6}{8.4}\selectfont\bfseries PSI}}%

    \vspace{0.5mm}%

    \noindent
    \tcbox[
      on line,enhanced,
      colback=TFSoftGray,colframe=TFSoftGray,boxrule=0pt,arc=1.2mm,
      width=0.50\linewidth,height=\TFValueH,
      left=0mm,right=0mm,top=0mm,bottom=0mm,
      valign=center,halign=center
    ]{{\fontsize{8.8}{9.8}\selectfont\bfseries #1}}%
    \tcbox[
      on line,enhanced,
      colback=TFSoftGray,colframe=TFSoftGray,boxrule=0pt,arc=1.2mm,
      width=0.50\linewidth,height=\TFValueH,
      left=0mm,right=0mm,top=0mm,bottom=0mm,
      valign=center,halign=center
    ]{{\fontsize{8.8}{9.8}\selectfont\bfseries #2}}%
  \end{tcolorbox}%
}

% =========================
% FILA DE FRENOS
% =========================
\newcommand{\TFBrakeRow}[4]{%
  \noindent
  \tcbox[on line,enhanced,colback=TFYellow,colframe=TFYellow,boxrule=0pt,arc=1.2mm,width=0.35\linewidth,height=\TFRowH,left=0.8mm,right=0.8mm,top=0.5mm,bottom=0.5mm,valign=center,halign=center]{{\fontsize{9.0}{10.2}\selectfont #1}}%
  \tcbox[on line,enhanced,colback=TFGreenBG,colframe=TFGreenBG,boxrule=0pt,arc=1.2mm,width=0.27\linewidth,height=\TFRowH,left=0.8mm,right=0.8mm,top=0.5mm,bottom=0.5mm,valign=center,halign=center]{{\fontsize{9.0}{10.2}\selectfont\bfseries\color{TFGreenText} #2}}%
  \tcbox[on line,enhanced,colback=TFYellow,colframe=TFYellow,boxrule=0pt,arc=1.2mm,width=0.20\linewidth,height=\TFRowH,left=0.8mm,right=0.8mm,top=0.5mm,bottom=0.5mm,valign=center,halign=center]{{\fontsize{9.0}{10.2}\selectfont #3}}%
  \tcbox[on line,enhanced,colback=TFGreenBG,colframe=TFGreenBG,boxrule=0pt,arc=1.2mm,width=0.18\linewidth,height=\TFRowH,left=0.8mm,right=0.8mm,top=0.5mm,bottom=0.5mm,valign=center,halign=center]{{\fontsize{9.0}{10.2}\selectfont\bfseries\color{TFGreenText} #4}}%
}

% =========================
% TARJETA COMPLETA
% =========================
\newcommand{\TFTireCard}[1]{%
  \par\noindent
  \begin{tcolorbox}[
    enhanced,
    colback=white,colframe=gray!30,boxrule=\TFCardBorder,arc=2.4mm,
    width=\TFCardW,left=\TFCardPad,right=\TFCardPad,top=\TFCardPad,bottom=\TFCardPad
  ]
    \TFHeader{Goodyear.png}{WRANGLER ALL TERRAIN ADVENTURE W/KEVLAR 110T}

    \vspace{0.5mm}%

    \begin{tcbraster}[
      raster columns=5,
      raster column skip=1.0mm,
      raster left skip=0mm,
      raster right skip=0mm
    ]
      \TFMetricCell{MEDIDA}{315/35/R21}
      \TFMetricCell{ÍNDICE DE\\VELOCIDAD}{210}
      \TFMetricCell{ÍNDICE DE\\CARGA}{210}
      \TFPSICell{35}{25}
      \TFMetricCell{PROFUNDIDAD\\DE BANDA (MM)}{3 mm}
    \end{tcbraster}

    \vspace{0.5mm}%

    \tcbox[
      enhanced,
      colback=TFHeadGray,colframe=TFHeadGray,boxrule=0pt,arc=1.2mm,
      width=\linewidth,height=\TFSectionH,
      left=0.8mm,right=0.8mm,top=0.4mm,bottom=0.4mm,
      valign=center,halign=center
    ]{{\fontsize{9.8}{11.2}\selectfont\bfseries FRENOS, DISCOS Y TAMBORES}}%

    \vspace{0.55mm}%
    \TFBrakeRow{Balatas/Tambores}{BUENA (>9MM )}{Desgaste}{8 mm}

    \vspace{0.5mm}%
    \TFBrakeRow{Discos}{BUENA (>9MM )}{Desgaste}{9 mm}
  \end{tcolorbox}%
  \par
}

\endinput

% =========================
% DOCUMENTO
% =========================
\begin{document}
%% page1.tex
%% (SOLO CONTENIDO — NO PONGAS \documentclass NI \usepackage AQUÍ)

% ===== HEADER =====
\noindent\includegraphics[width=\textwidth]{top.png}
\vspace{0mm}

% ===== TÍTULO =====
{\centering
  \fontsize{18}{22}\selectfont \bfseries
  MULTI-PUNTOS DE INSPECCIÓN (PRE-SERVICIO)\par
}

% Ajusta este valor para la distancia con "Información General"
\vspace{1.5mm}

% ===== BLOQUE GRIS: INFORMACIÓN GENERAL =====
\noindent\makebox[\textwidth][c]{%
  \begin{tcolorbox}[
    colback=gray!25,
    colframe=gray!25,
    width=\contentwidth,
    arc=1mm,
    boxrule=0pt,
    left=\padlr,
    right=\padlr,
    top=1mm,
    bottom=1mm
  ]
    \centering
    {\fontsize{15}{18}\selectfont \bfseries Información General}
  \end{tcolorbox}%
}

% ===== CONTENIDO =====
\begin{center}
  \begin{minipage}{\contentwidth}
    \begin{minipage}{\innerwidth}

      \vspace{1mm}

      % Subtítulo (Kode Mono)
      {\kodemono\fontseries{b}\fontsize{12}{14}\selectfont Información General del Vehículo}
      \vspace{2mm}

      % --- FILA 1 ---
      \begin{tabularx}{\textwidth}{X @{\hspace{3mm}} p{0.22\textwidth} @{\hspace{3mm}} p{0.18\textwidth}}
        \fieldlabel{Modelo del vehículo} & \fieldlabel{Placas} & \fieldlabel{Kilometraje} \\
        \begin{inputbox}\fontsize{10.5}{12}\selectfont VOLKSWAGEN TIGUAN 2026\end{inputbox} &
        \begin{inputbox}\fontsize{10.5}{12}\selectfont JCZ-263-A\end{inputbox} &
        \begin{inputbox}\fontsize{10.5}{12}\selectfont 85,000\end{inputbox} \\
      \end{tabularx}

      \vspace{2mm}

      % --- FILA 2 ---
      \begin{tabularx}{\textwidth}{p{0.30\textwidth} @{\hspace{3mm}} X}
        \fieldlabel{Número de serie} & \fieldlabel{Razón de ingreso} \\
        \begin{inputbox}\fontsize{10.5}{12}\selectfont 3N1BC13E38L592153\end{inputbox} &
        \begin{inputbox}\fontsize{10.5}{12}\selectfont CAMBIO DE LLANTAS CON ALINEACIÓN Y BALANCEO\end{inputbox} \\
      \end{tabularx}

      % ===== NUEVA SECCIÓN: LUCES DE ADVERTENCIA =====
      \vspace{3mm}
      {\fontsize{10}{12}\selectfont Luces de advertencia encendidas}
      \vspace{2mm}

      % --- Chips ---
      \chip{Batería}\hspace{2mm}
      \chip{Cinturón}\hspace{2mm}
      \chip{Puertas abiertas}

      \vspace{4mm}

      % --- Grid ---
      \begin{tikzpicture}[x=1cm,y=1cm]
        % FILA 1
        \warniconij{0}{0}{engine.png}{1}
        \warniconij{1}{0}{oil.png}{0}
        \warniconij{2}{0}{battery.png}{1}
        \warniconij{3}{0}{air.png}{0}
        \warniconij{4}{0}{throttle.png}{0}
        \warniconij{5}{0}{abs.png}{0}
        \warniconij{6}{0}{tractionControl.png}{1}
        \warniconij{7}{0}{doors.png}{1}
        \warniconij{8}{0}{immobilizer.png}{0}
        \warniconij{9}{0}{brake.png}{0}
        \warniconij{10}{0}{tire.png}{0}
        \warniconij{11}{0}{brakePark.png}{1}

        % FILA 2
        \warniconij{0}{1}{celsius.png}{0}
        \warniconij{1}{1}{maintenance.png}{0}
        \warniconij{2}{1}{lighbulb.png}{0}
        \warniconij{3}{1}{steering-wheel.png}{0}
        \warniconij{4}{1}{windshield-washer.png}{1}
      \end{tikzpicture}

      % ===== NUEVA SECCIÓN: NIVEL DE GASOLINA =====
      \vspace{3mm}
      {\fontsize{10}{12}\selectfont Nivel de gasolina}
      \vspace{3mm}

      \gasolinebar{4} % 0=BAJO, 1=1/4, 2=MEDIO, 3=3/4, 4=LLENO

    \end{minipage}
  \end{minipage}
\end{center}

% ===== BLOQUE GRIS: LÍQUIDOS =====
\vspace{2mm}
\noindent\makebox[\textwidth][c]{%
  \begin{tcolorbox}[
    colback=gray!25,
    colframe=gray!25,
    width=\contentwidth,
    arc=1mm,
    boxrule=0pt,
    left=\padlr,
    right=\padlr,
    top=1mm,
    bottom=1mm
  ]
    \centering
    {\fontsize{15}{18}\selectfont \bfseries Líquidos}
  \end{tcolorbox}%
}

% ===== CONTENIDO LÍQUIDOS =====
\begin{center}
  \begin{minipage}{\contentwidth}
    \begin{minipage}{\innerwidth}

      \vspace{1.5mm}
      {\kodemono\fontseries{b}\fontsize{12}{14}\selectfont Inspección de Líquidos}
      \vspace{1.5mm}

      \noindent
      \begin{minipage}[t]{0.623\innerwidth}
        {\fontsize{9}{11}\selectfont Nivel de aceite}
        \vspace{0.8mm}

        \noindent
        \oillevelbar{2}
      \end{minipage}%
      \hspace{3mm}%
      \begin{minipage}[t]{\dimexpr 0.377\innerwidth-3mm\relax}
        {\fontsize{9}{11}\selectfont Color de aceite}
        \vspace{0.8mm}

        \noindent
        \oilcolorselector{2}
      \end{minipage}

      \vspace{4mm}

      \noindent
      \begin{minipage}[t]{0.32\innerwidth}
        \fluidselectorfour{1}{Anticongelante}
      \end{minipage}\hfill
      \begin{minipage}[t]{0.32\innerwidth}
        \fluidselectorfour{3}{Dirección Hidraulica}
      \end{minipage}\hfill
      \begin{minipage}[t]{0.32\innerwidth}
        \fluidselectorfour{0}{Líquido de frenos}
      \end{minipage}

      \vspace{4mm}

      % --- Comentarios + Limpia parabrisas (como screenshot) ---
      \noindent
      \begin{minipage}[t]{0.70\innerwidth}
        {\fontsize{9}{11}\selectfont Comentarios adicionales}
        \vspace{1mm}

        \begin{commentbox}
        % (en blanco, listo para texto)
        \end{commentbox}
      \end{minipage}\hfill
      \begin{minipage}[t]{0.27\innerwidth}
        {\fontsize{9}{11}\selectfont Líquido limpiaparabrisas}
        \vspace{1mm}

        \yesnoselector{0} % 0=SI, 1=NO
      \end{minipage}

    \end{minipage}
  \end{minipage}
\end{center}

\newpage
%% page2.tex
%% Layout completo de llantas y frenos (2x2 + carro al centro)

% ============================================================
% CONTROLES GENERALES DE ESTA PÁGINA
% ============================================================
\newcommand{\TFGlobalScale}{0.30}      % 70% menos (queda al 30%)
\newcommand{\TFBaseCanvasW}{3.2\textwidth} % ancho base antes de escalar
\newcommand{\TFPageTopGap}{2mm}
\newcommand{\TFColumnsGapTop}{5mm}
\newcommand{\TFCenterImageWidth}{0.98\linewidth}
\newcommand{\TFMainBarHeight}{9mm}

\vspace*{\TFPageTopGap}

\noindent\makebox[\textwidth][c]{%
\scalebox{\TFGlobalScale}{%
\begin{minipage}{\TFBaseCanvasW}

% Barra de título principal
\noindent\makebox[\textwidth][c]{%
  \tcbox[
    colback=gray!25,colframe=gray!25,boxrule=0pt,arc=2mm,
    width=0.94\textwidth,height=\TFMainBarHeight,valign=center,halign=center,
    left=0mm,right=0mm,top=0mm,bottom=0mm
  ]{{\fontsize{24}{26}\selectfont\bfseries Llantas y Frenos}}%
}

\vspace{\TFColumnsGapTop}

% Bloque principal: izquierda / carro / derecha
\noindent
\begin{minipage}[t]{0.39\textwidth}
  \TFTitle{Delantero Derecha}
  \TFTireCard{FR}

  \vspace{4.6mm}

  \TFTitle{Trasera Derecha}
  \TFTireCard{RR}
\end{minipage}
\hfill
\begin{minipage}[t]{0.20\textwidth}
  \vspace{8mm}
  \centering\includegraphics[width=\TFCenterImageWidth]{carro.png}
\end{minipage}
\hfill
\begin{minipage}[t]{0.39\textwidth}
  \TFTitle{Delantera Izquierda}
  \TFTireCard{FL}

  \vspace{4.6mm}

  \TFTitle{Trasera Izquierda}
  \TFTireCard{RL}
\end{minipage}

\end{minipage}%
}%
}

\newpage
%% page3.tex
%% (SOLO CONTENIDO — NO PONGAS \documentclass NI \usepackage AQUÍ)

% =========================
% Página 3: Amortiguadores y Bases
% =========================

\newcommand{\ShockTitleBar}[1]{%
  \noindent\makebox[\textwidth][c]{%
    \begin{tcolorbox}[
      colback=gray!25,
      colframe=gray!25,
      width=\contentwidth,
      arc=1mm,
      boxrule=0pt,
      left=\padlr,
      right=\padlr,
      top=1mm,
      bottom=1mm
    ]
      \centering
      {\fontsize{15}{18}\selectfont \bfseries #1}
    \end{tcolorbox}%
  }%
}

\newcommand{\ShockGroupHeader}[1]{%
  \begin{tcolorbox}[
    enhanced,
    nobeforeafter,
    width=\linewidth,
    height=7mm,
    colback=warningyellow,
    colframe=warningyellow,
    boxrule=0pt,
    arc=1.3mm,
    left=0mm,right=0mm,top=0mm,bottom=0mm,
    boxsep=0mm,
    valign=center,halign=center
  ]
    {\fontsize{10.8}{11}\selectfont\bfseries #1}
  \end{tcolorbox}%
}

\newcommand{\ShockStateHeaderCell}[2]{%
  \begin{tcolorbox}[
    enhanced,
    nobeforeafter,
    width=\linewidth,
    height=4.3mm,
    colback=#1,
    colframe=#1,
    boxrule=0pt,
    arc=0mm,
    left=0mm,right=0mm,top=0mm,bottom=0mm,
    boxsep=0mm,
    valign=center,halign=center
  ]
    {\color{white}\fontsize{7}{7.6}\selectfont\bfseries #2}
  \end{tcolorbox}%
}

\newcommand{\ShockRowLabel}[1]{%
  \begin{tcolorbox}[
    enhanced,
    nobeforeafter,
    width=\linewidth,
    height=7.2mm,
    colback=warningyellow,
    colframe=warningyellow,
    boxrule=0pt,
    arc=1.3mm,
    left=0mm,right=0mm,top=0mm,bottom=0mm,
    boxsep=0mm,
    valign=center,halign=center
  ]
    {\fontsize{9.8}{10}\selectfont\bfseries #1}
  \end{tcolorbox}%
}

\newcommand{\ShockBlankCell}{%
  \begin{tcolorbox}[
    enhanced,
    nobeforeafter,
    width=\linewidth,
    height=7.2mm,
    colback=gray!12,
    colframe=gray!12,
    boxrule=0pt,
    arc=1.3mm,
    left=0mm,right=0mm,top=0mm,bottom=0mm,
    boxsep=0mm,
    valign=center,halign=center
  ]\end{tcolorbox}%
}

\newcommand{\ShockDotCell}[1]{%
  \begin{tcolorbox}[
    enhanced,
    nobeforeafter,
    width=\linewidth,
    height=7.2mm,
    colback=gray!12,
    colframe=gray!12,
    boxrule=0pt,
    arc=1.3mm,
    left=0mm,right=0mm,top=0mm,bottom=0mm,
    boxsep=0mm,
    valign=center,halign=center
  ]
    \tikz\fill[#1] (0,0) circle (2.15mm);
  \end{tcolorbox}%
}

\ShockTitleBar{Amortiguadores y Bases}

\vspace{1.6mm}

\noindent\makebox[\textwidth][c]{%
\begin{minipage}{\contentwidth}
  {\kodemono\fontseries{sb}\fontsize{13}{14}\selectfont Inspección de Amortiguadores y Bases Delanteras}
\end{minipage}%
}

\vspace{2.4mm}

\noindent\makebox[\textwidth][c]{%
\begin{minipage}{\contentwidth}
  \setlength{\tabcolsep}{0pt}
  \renewcommand{\arraystretch}{1.05}
  \begin{tabular}{@{}p{0.18\linewidth}@{\hspace{1mm}}p{0.125\linewidth}@{\hspace{1mm}}p{0.125\linewidth}@{\hspace{1mm}}p{0.125\linewidth}@{\hspace{3mm}}p{0.125\linewidth}@{\hspace{1mm}}p{0.125\linewidth}@{\hspace{1mm}}p{0.125\linewidth}@{}}
    & \multicolumn{3}{c}{\ShockGroupHeader{AMORTIGUADORES}} & \multicolumn{3}{c}{\ShockGroupHeader{BASES}} \\
    & \ShockStateHeaderCell{green!65!black}{BUENA} & \ShockStateHeaderCell{warningyellow}{REGULAR} & \ShockStateHeaderCell{red!85!black}{MALA} & \ShockStateHeaderCell{green!65!black}{BUENA} & \ShockStateHeaderCell{warningyellow}{REGULAR} & \ShockStateHeaderCell{red!85!black}{MALA} \\

    \ShockRowLabel{DELANTERA D.} & \ShockDotCell{green!65!black} & \ShockBlankCell & \ShockBlankCell & \ShockBlankCell & \ShockBlankCell & \ShockBlankCell \\
    \ShockRowLabel{DELANTERA I.} & \ShockBlankCell & \ShockDotCell{warningyellow} & \ShockBlankCell & \ShockBlankCell & \ShockBlankCell & \ShockBlankCell \\
    \ShockRowLabel{TRASERA D.} & \ShockBlankCell & \ShockBlankCell & \ShockDotCell{red!85!black} & \ShockBlankCell & \ShockBlankCell & \ShockBlankCell \\
    \ShockRowLabel{TRASERA I.} & \ShockDotCell{green!65!black} & \ShockBlankCell & \ShockBlankCell & \ShockBlankCell & \ShockBlankCell & \ShockBlankCell \\
  \end{tabular}
\end{minipage}%
}

\end{document}
